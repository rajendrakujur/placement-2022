\chapter{Complexity Classes} 
A complexity class consists of a set of problems which have the same rate of growth of a
particular resource as the input size of the problem increases. The resource which we are
concerned with here is time. Problems are decided to belong to a particular complexity class
by running it on a Turing machine.
The temporary storage of a Turing machine is a tape. The tape is divided into cells,
each cell contains a symbol from the tape alphabet. A read-write head moves along the cells
of the tape and can read or write a symbol at a time. The symbols in the tape at the starting
point form the input and the symbols in the tape in a final state where the machine halts form
the output. The read-write head on seeing a symbol in a particular state moves to another
state by changing or retaining the symbol, this is done according to the transition function.
A Turing machine can be either deterministic or non deterministic. In a non deterministic
Turing Machine, the machine can move from one particular state to more than one states
non deterministically. We now recall the definitions of a few complexity class

\section{P}
\noindent
P can be described as a complexity class which involves collection of all decision problems
that can be solved in polynomial time. That is, the answer yes or no for a given decision can
be decided in polynomial time.A problem belongs to the complexity class P if there exists
a deterministic Turing machine M, which runs for polynomial time on all valid inputs of
the problem and halts in a final state.Finding maximum of given array of elements , finding
lowest common multiple of two numbers, finding a dominating set in a tree, are examples of
a some problems which belong to P class.
\section{NP}
\noindent
NP can be described as a complexity class which involves the collection of all
decision problems for which if given an answer,it can be verified correct or not in
polynomial time.A problem belongs to the complexity class NP if there exists a non
deterministic Turing machine M, which runs for polynomial time on all valid inputs of
the problem and halts in a final state. NP stands for non deterministic polynomial time.
Given a solution of such a problem, it can be verified efficiently in polynomial time by a
deterministic Turing machine.
\section{NP-complete}
\noindent
NP-Complete is a complexity class which represents the set of all problems X in NP
for which it is possible to reduce every other NP problem Y to X in polynomial
time.Intuitively this means that we can solve Y quickly if we know how to solve
X quickly. Precisely,Y is reducible to X, if there is a polynomial time algorithm f
to transform instances y of Y to instances x = f (y) of X in polynomial time, with the
property that the answer to y is yes, if and only if the answer to f (y) is yes.
\section{NP-Hard}
\noindent

Intuitively, these are the problems that are at least as hard as
the NP-complete problems.Note that NP-hard problems do not have to be in NP,
and they do not have to be decision problems.The precise definition here is that
a problem X is NP-hard, if there is an NP-complete problem Y, such that Y is reducible
to X in polynomial time.But since any NP-complete problem can be reduced to any
other NP-complete problem in polynomial time, all NP-complete problems can be
reduced to any NP-hard problem in polynomial time. Then,if there is a solution to
one NP-hard problem in polynomial time, there is solution to all NP problems in
polynomial time.